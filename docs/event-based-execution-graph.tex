\documentclass[11pt,a4paper]{article}
\usepackage[hscale=0.83,vscale=0.92]{geometry}
\usepackage[utf8]{inputenc}
\usepackage{array}
\usepackage{booktabs}
\usepackage[font=footnotesize,labelfont=bf]{caption}
\usepackage{amsmath}
\usepackage{amsfonts}
\usepackage{amssymb}
\usepackage{algorithm}
\usepackage{algorithmic}
\usepackage{graphicx}
\usepackage{hyperref}
\graphicspath{{./analysis/}}

\newcommand{\ord}{\textsuperscript{th} }
\newcommand{\code}[1]{\texttt{#1} }

\title{OTTER's Event-Based Execution Graph}
\begin{document}

\author{Adam Tuft}
\maketitle

\section{Events}

\begin{itemize}
\item An event records the information required to describe a node in the execution graph.
\item Events are created in the order that they are encountered by a thread.
\item OTTER events are isomorphic to a subset of OMP execution model events.
\item Each thread maintains a queue of events it encounters.
\item A new event is created for each \emph{scope-begin}, \emph{scope-end} and \emph{synchronisation-end} event.
\item OMP constructs associated with structured blocks, that generate \emph{scope-begin} and \emph{scope-end} events represented by OTTER events:
\begin{itemize}
\item Parallel regions
\item Single regions
\item Sections regions
\item Loops
\item Taskloops
\item Taskgroups
\end{itemize}
\item Standalone OMP synchronisation constructs generating \emph{synchronisation-end} events that are represented by OTTER events:
\begin{itemize}
\item Barriers (explicit, implicit \& implementation)
\item Taskwait constructs
\end{itemize}
\item At a task synchronisation event (e.g. a \code{taskwait}), the event receives references to all relevant task-create events (from the encountering task) generated since the most recent task synchronisation event in that thread (but \textbf{must not} receive task-create events that occur \textbf{after} the synchronisation event).
\end{itemize}

\section{Tasks}

\begin{itemize}
\item \emph{task-create} events are generated for each explicit generated task.
\item A \emph{task-create} event references the data for the created task.
\item A task records the \emph{task-create} events for which it is the encountering task.
\item When an implicit task receives a \emph{task-create} event, that event is also added to the encountering thread's event queue.
\item A task \textbf{does not} record the execution event which immediately precedes its \emph{task-create} event
\item A task inherits its parent event from its parent task at the point it is created.
\item When it generates a task, an implicit task supplies the most recent event recorded by its thread to the generated task as its parent event.
\item Each task records the child tasks it must hand off to the next task synchronisation construct it encounters
\item At a task synchronisation event (e.g. taskwait, taskgroup), the \textbf{encountering} task hands off to the event a collection of tasks that are synchronised by the task synchronisation construct.
\end{itemize}

\end{document}
