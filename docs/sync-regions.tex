\documentclass[11pt,a4paper]{article}
\usepackage[hscale=0.83,vscale=0.92]{geometry}
\usepackage[utf8]{inputenc}
\usepackage{array}
\usepackage{booktabs}
\usepackage[font=footnotesize,labelfont=bf]{caption}
\usepackage{amsmath}
\usepackage{amsfonts}
\usepackage{amssymb}
\usepackage{algorithm}
\usepackage{algorithmic}
\usepackage{graphicx}
\usepackage{hyperref}
\graphicspath{{./analysis/}}

\newcommand{\ord}{\textsuperscript{th} }
\newcommand{\code}[1]{\texttt{#1} }

\title{OMP Syc Regions}
\begin{document}

\author{Adam Tuft}
\maketitle

\section{Barriers}

\section{Taskwait}

Specifies a wait on the completion of child tasks of the current (encountering) task. Binds to the current task region.

Effect = current task suspended at taskwait until all child tasks generated before the taskwait are complete.

If depend clause(s) present, acts as if an empty task was created wth same depend clause(s) that is both \emph{mergeable} (may be \emph{merged} if also \emph{undeferred} or \emph{included}) and \emph{included} (\emph{undeferred} and executed by encountering thread). Effectively, taskwait \& depend means wait until predecessor tasks of the depend(s) are complete.

A \emph{merged} task has the same data environment as its generating task region.

An \emph{undeferred} task causes its encountering task to be suspended until the structured block of the \emph{undeferred} task is completed.

\subsection{Required Behaviour}

\begin{itemize}
\item A taskwait node is a child of its encountering task's explicit child task nodes
\item If the encountering task is implicit, a taskwait node is a child of the encountering scope's explicit child task nodes
\item A taskwait node is a child of the prior scope node and an ancestor of the following scope node (either of which may be scope-begin or scope-end)
\item When a team of threads encounters a taskwait construct, all threads share a reference to the same taskwait node when creating edges between the encountering tasks and the taskwait node.
\end{itemize}

At a taskwait barrier, the first thread to arrive creates the taskwait node and shares a reference to this node among all threads at the barrier (via the shared parallel data). Later threads read the reference and create edges from their encountering task to the taskwait node as approptiate (see above). The master thread connects the taskwait node to the prior scope node and ensures that subsequent nodes can connect to the taskwait node. The last thread to arrive resets the shared parallel data.

\section{Taskgroup}

\section{Reduction}

\end{document}
