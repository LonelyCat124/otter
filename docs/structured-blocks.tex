\documentclass[11pt,a4paper]{article}
\usepackage[hscale=0.83,vscale=0.92]{geometry}
\usepackage[utf8]{inputenc}
\usepackage{array}
\usepackage{booktabs}
\usepackage[font=footnotesize,labelfont=bf]{caption}
\usepackage{amsmath}
\usepackage{amsfonts}
\usepackage{amssymb}
\usepackage{algorithm}
\usepackage{algorithmic}
\usepackage{graphicx}
\usepackage{hyperref}
\graphicspath{{./analysis/}}

\newcommand{\ord}{\textsuperscript{th} }
\newcommand{\code}[1]{\texttt{#1} }

\title{OMP Structured Block Constructs (Scopes)}
\begin{document}

\author{Adam Tuft}
\maketitle

The master thread of a team owns a mutex on the prior node of the current team. A thread that does not hold this lock \textbf{must not} attempt to access the prior node. At \emph{single-begin-other} the master thread releases the lock on the prior node as it does not execute the structured block of the single region. At \emph{single-begin} the encountering thread (if it is not the master thread) acquires the lock on the prior node and owns the lock for the duration of the single region. At \emph{single-end} it releases the lock (if it is not the master thread). At \emph{single-end-other} the master thread re-acquires the lock on the prior node, which should now store the latest node created by the encountering thead during the single region.

At \emph{scope-begin}, \emph{scope-end} and \emph{sync-end} the master thread of a team creates the required graph node and connects it to the prior node (may be initial task, scope-begin, scope-end or sync-region node). It stores the node as the prior node in the enclosing parallel scope (for all threads to access as needed).

At \emph{single-begin} , the encountering thread becomes responsible for creating graph nodes and storing the prior node in the parallel scope. This is required so that the master thread can obtain the \emph{single-end} node immediately at the following node (as there will not be an implicit barrier node if a \code{nowait} is present)

\section{Connecting Nested \& Consecutive Scopes}

\begin{itemize}
\item Master thread detects prior scope and connects new scope
\end{itemize}

\section{Synchronisation Nodes}

At \emph{sync-end} the first thread to arrive creates the sync node and posts a reference to it in the enclosing parallel scope. The master thread connects the prior node to the sync node. Each thread stores the node as their prior node.

This allows the threads to create any necessary edges to the same synchronisation node.

\end{document}
